% Options for packages loaded elsewhere
\PassOptionsToPackage{unicode}{hyperref}
\PassOptionsToPackage{hyphens}{url}
%
\documentclass[
]{article}
\usepackage{lmodern}
\usepackage{amssymb,amsmath}
\usepackage{ifxetex,ifluatex}
\ifnum 0\ifxetex 1\fi\ifluatex 1\fi=0 % if pdftex
  \usepackage[T1]{fontenc}
  \usepackage[utf8]{inputenc}
  \usepackage{textcomp} % provide euro and other symbols
\else % if luatex or xetex
  \usepackage{unicode-math}
  \defaultfontfeatures{Scale=MatchLowercase}
  \defaultfontfeatures[\rmfamily]{Ligatures=TeX,Scale=1}
\fi
% Use upquote if available, for straight quotes in verbatim environments
\IfFileExists{upquote.sty}{\usepackage{upquote}}{}
\IfFileExists{microtype.sty}{% use microtype if available
  \usepackage[]{microtype}
  \UseMicrotypeSet[protrusion]{basicmath} % disable protrusion for tt fonts
}{}
\makeatletter
\@ifundefined{KOMAClassName}{% if non-KOMA class
  \IfFileExists{parskip.sty}{%
    \usepackage{parskip}
  }{% else
    \setlength{\parindent}{0pt}
    \setlength{\parskip}{6pt plus 2pt minus 1pt}}
}{% if KOMA class
  \KOMAoptions{parskip=half}}
\makeatother
\usepackage{xcolor}
\IfFileExists{xurl.sty}{\usepackage{xurl}}{} % add URL line breaks if available
\IfFileExists{bookmark.sty}{\usepackage{bookmark}}{\usepackage{hyperref}}
\hypersetup{
  pdftitle={Rapport},
  hidelinks,
  pdfcreator={LaTeX via pandoc}}
\urlstyle{same} % disable monospaced font for URLs
\setlength{\emergencystretch}{3em} % prevent overfull lines
\providecommand{\tightlist}{%
  \setlength{\itemsep}{0pt}\setlength{\parskip}{0pt}}
\setcounter{secnumdepth}{-\maxdimen} % remove section numbering

% Left margins
\usepackage{marginnote}
\newcommand{\Question}[1]{\reversemarginpar\marginnote{\textbf{Q#1.}}}

\usepackage[ruled, french, frenchkw]{algorithm2e} 
\usepackage{mathtools}


\DeclarePairedDelimiter\abs{\lvert}{\rvert}%
\DeclarePairedDelimiter\norm{\lVert}{\rVert}%
\DeclarePairedDelimiter\ceil{\lceil}{\rceil}%
\DeclarePairedDelimiter\floor{\lfloor}{\rfloor}%

\DeclareMathOperator*{\card}{card}%
\DeclareMathOperator*{\argmin}{argmin}%
\DeclareMathOperator*{\Mat}{Mat}%

\newcommand{\N}{\mathbb{N}}
\newcommand{\R}{\mathbb{R}}
\newcommand{\Z}{\mathbb{Z}}
\newcommand{\M}{\mathcal{M}}
\newcommand{\class}[1]{\mathcal{C}^{#1}}

% Swap the definition of \abs* and \norm*, so that \abs
% and \norm resizes the size of the brackets, and the 
% starred version does not.
\makeatletter
\let\oldabs\abs
\def\abs{\@ifstar{\oldabs}{\oldabs*}}
%
\let\oldnorm\norm
\def\norm{\@ifstar{\oldnorm}{\oldnorm*}}
\makeatother

\title{Rapport}
\date{}

\begin{document}
\maketitle

\section{Introduction}

En 1966, de Bruijn lance le projet Automath qui a pour visée de pouvoir exprimer des théories mathématiques complètes, c'est-à-dire des théories qui sont des ensembles maximaux cohérents de propositions, i.e. le théorème d'incomplétude de Gödel ne s'y applique pas notamment.

Peu après, les projets Mizar, HOL-Isabelle et Coq naissent pour devenir les assistants de preuve mathématiques que l'on connaît.

Ces projets mettent à disposition un ensemble d'outil afin d'aider le mathématicien à formaliser sa preuve dans une théorie mathématiques de son choix: ZFC, la théorie des types dépendants, la théorie des types homotopiques par exemple.

Certains assistants de preuve ne se contentent pas de vérifier la formalisation d'une preuve mais peuvent aussi effectuer de la décision (dans l'arithmétique de Presburger par exemple).

L'enjeu des assistants de preuve et des concepts utilisés derrière dépasse le simple outil de mathématicien.
D'une part, ils permettent d'attaquer des problèmes qui ont résisté pendant longtemps, le théorème des quatre couleurs par exemple.
D'autre part, leurs usages se généralisent afin de pouvoir faire de la certification informatique, démontrer qu'un programme vérifie un certain nombre d'invariants, par exemple, dans l'aviation, des outils similaires sont employés pour certifier le comportement de certaines pièces embarquées.

\section{Détail du Number Games}

\end{document}
