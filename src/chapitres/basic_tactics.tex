\section{Tactics de base}

\paragraph{intro}

Parfois, le but que nous cherchons à atteindre est une implication. Pour prouver que 'A -> B', on va prouver 'B sachant A vrai'. En Lean, cela revient à inclure A dans les hypothèses et à changer le but en 'B'. C'est ce que fait la tactic 'intro'. On peut donner un nom à l'hypothèse qu'on introduit : 'intro h,' ou laisser Lean choisir un nom par défaut.
On peut écrire 'intros h1 h2 ... hn,' pour introduire plusieurs hypothèses en même temps.
On entrera plus en détails dans la structure des implications dans le Function World.

\paragraph{have}

Pour déclarer une nouvelle hypothèse, on peut utiliser la tactic 'have'.
'have p : P' diviser le but en 2 sous-but : montrer qu'on peut construire un élément de type P avec les hypothèses actuelles puis montrer le but initial avec l'hypothèse 'p : P' en plus.
Lorsque la preuve de l'existence de l'objet qu'on crée est brève, on peut contracter sa définition :
'have p := f a' avec 'a : A' et 'f : A -> P' comme hypothèses déjà présentes ajoutera directement 'p : P' dans la liste d'hypothèses.  

\paragraph {\large \textbf{refl}}: Cette tactique correspond à la réflexivité de l'égalité, d'où le nom \textbf{refl}. Elle peut s'appliquer pour prouver toute égalité de la forme $A=A$. C'est à dire, toute égalité dont les deux membres sont égaux terme à terme. \\
\textit{Exemple}Ssoient $x,y,z,w$ des entiers naturels, alors on peut prouver que $x+y \times (z+w)=x+y \times (z+w)$ en appliquant la tactique \big\{\textbf{refl,}\big\}.
\paragraph {\large\textbf{rw}}: Le nom de cette tactique correspond au mot anglais \textit{rewrite}. Elle s'applique dans deux cas distincts.

Soit $H : A = B$ une hypothèse d'égalité. \footnote{ou une preuve de $A = B$, c'est la même chose.}
Supposons que l'équation à démontrer est le mot $F$. \\Si $F$ contient au moins un $A$, l'instruction \big\{\textbf{rw $H$,}\big\} dérive un mot $F'$ du mot $F$, en effectuant un seul changement: tous les $A$ (présents dans $F$) sont réécrits en $B$. De même, si $F$ contient au moins un $B$ et si on utilise \big\{\textbf{rw $\leftarrow$ H,}\big\}, alors le seul changement sera: tous les $B$ (présents dans $F$) sont réécrits en $As$.

Soit $T: A=B$, c'est à dire $T$ est une preuve de $A=B$, supposé faite à un niveau qui précède le niveau traité. Dans ce cas, elle figure sur le menu des théorèmes. Alors \big\{\textbf{rw T,}\big\} (respectivement \big\{\textbf{rw $\leftarrow$ T,}\big\}) dérive un mot $F'$ du mot $F$, en effectuant un seul changement: tous les $As$ (resp. $Bs$) sont remplacés par des $Bs$ (resp. $As$). 
\paragraph {\large\textbf{simp}}: C'est une tactique de haut niveau. Elle est disponible à partir du dernier niveau de \textit{Addition World}. Son principe est le suivant: elle utilise la tactique \textbf{rw} avec les preuves des théorèmes d'associativité et de commutativité de l'addition pour prouver une certaine égalité (les preuves de l' associativité et la commutativité de la multiplication sont disponibles à partir du dernier niveau de \textit{Multiplication World}). De plus, à l'aide du langage de métaprogrammation de Lean, on peut éventuellement apprendre  à \textbf{simp} à simplifier une variété de formules plus large en utilisant d'autres preuves outre celles de l'associativité et de la commutativité.   \\
\textit{Exemple} Soient $x,y,z,w,u$ des entiers naturels, alors on peut démontrer que $x+y+z+w+u=y+(z+x+u)+w$ en utilisant \big\{\textbf{simp,} \big\}
