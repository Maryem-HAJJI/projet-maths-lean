\section{Tactics de base}

\paragraph{intro}

Parfois, le but que nous cherchons à atteindre est une implication. Pour prouver que 'A -> B', on va prouver 'B sachant A vrai'. En Lean, cela revient à inclure A dans les hypothèses et à changer le but en 'B'. C'est ce que fait la tactic 'intro'. On peut donner un nom à l'hypothèse qu'on introduit : 'intro h,' ou laisser Lean choisir un nom par défaut.
On peut écrire 'intros h1 h2 ... hn,' pour introduire plusieurs hypothèses en même temps.
On entrera plus en détails dans la structure des implications dans le Function World.

\paragraph{have}

Pour déclarer une nouvelle hypothèse, on peut utiliser la tactic 'have'.
'have p : P' diviser le but en 2 sous-but : montrer qu'on peut construire un élément de type P avec les hypothèses actuelles puis montrer le but initial avec l'hypothèse 'p : P' en plus.
Lorsque la preuve de l'existence de l'objet qu'on crée est brève, on peut contracter sa définition :
'have p := f a' avec 'a : A' et 'f : A -> P' comme hypothèses déjà présentes ajoutera directement 'p : P' dans la liste d'hypothèses.  
