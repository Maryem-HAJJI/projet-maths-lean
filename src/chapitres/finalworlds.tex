\paragraph{Advanced Addition World : Level 10}

Ce lemme ressemble à la régularité à gauche du monoïde $(\mynat, +)$ qu'on a prouvé au niveau 6 de ce monde.
On a démontré que :
\begin{minted}[mathescape,linenos,numbersep=5pt,frame=lines,framesep=2mm,breaklines,escapeinside=||]{lean}
(a b c : mynat) : a + b = a + c → b = c
\end{minted}
Donc pour $a = 0$ et $c = 0$ et $a + b = a + c$ impliquent $b = 0$.
Dans ce lemme, nous allons montrer qu'il suffit d'avoir les hypothèses $a+c = 0$ et $c = 0$ et $a + b = a + c$ pour prouver $b = 0$.

\begin{minted}[mathescape,linenos,numbersep=5pt,frame=lines,framesep=2mm,breaklines,escapeinside=||]{lean}
lemma add_left_eq_zero {{a b : mynat}} : a + b = 0 → b = 0 :=
begin [nat_num_game]
  intro H,
  --On fait une distinction de cas
  --Soit $b = 0$ soit il existe d : mynat tel que $b = \succ(d)$ :
  cases b with d,

  --Cas $b = 0$, le but devient $0 = 0$, la résolution est triviale :
  refl,

  --Cas $b = \succ(d)$, le but devient $\succ d = 0$
  --Ce qui est impossible, cela contredit l'axiome de Peano $zero_ne_succ$
  --On va donc faire une preuve par l'absurde :
  rw add_succ at H, --on fait rentrer a dans le succ donc $H : \succ (a + d) = 0$
  exfalso, -- le but est impossible à prouver donc on le change en $faux$
  --Et on a le théorème succ_ne_zero $(n : mynat) : \succ n = 0 -> faux$, donc on sait que l'hypothèse $H$ implique $faux$
  exact succ_ne_zero H,
  --On a donc prouver que l'hypothèse $\exists d : mynat tel que b = \succ(d)$ est est contradictoire avec nos axiome
end
\end{minted}

\paragraph{Advanced Multiplication World : Level 4}

Ce théorème consiste à prouver la régularité à gauche du monoïde (\mynat, *).
L'idée est instinctive mais la preuve nécessite en réalité beaucoup de distinctions de cas et l'utilisation d'une nouvelle tactique, \textit{revert}.
\begin{minted}[mathescape,linenos,numbersep=5pt,frame=lines,framesep=2mm,breaklines,escapeinside=||]{lean}
theorem mul_left_cancel (a b c : mynat) (ha : a ≠ 0) : a * b = a * c → b = c :=
begin
  revert b,
  --On ne considère plus $b$ comme une hypothèse,
  --à la place, on rajoute un $\forall (b : mynat)$ au but
  --Ce sera utile plus tard, dans l'hypothèse d'induction
  
  -- On fait une induction sur $c$ :
  induction c with n hn,

  --Le cas de base $\forall (b : mynat), a * b = a * 0 → b = 0$ :
  rw mul_zero, --On simplifie
  intros b h, --On introduit un $b$ et l'hypothèse $h : a * b = 0$
  rw mul_eq_zero_iff a b at h, --$h$ est équivalent à $a = 0 or b = 0$ donc on la réécrit
  
  --On casse le $a = 0 or b = 0$ en deux cas:
  cases h with hha hhb,
  
  --Si $a = 0$ (but : $b = 0$) :
  --On a $a \ne 0$ en hypothèse donc on sait que ce cas est impossible
  --On va donc faire une preuve par l'absurde :
  exfalso, --but = $false$
  apply ha, --but = $a = 0$
  exact hha, --Il n'y a plus qu'à appliquer l'hypothèse de disjonction de cas
  
  --Si $b = 0$ (but : $b = 0$), c'est trivial :
  exact hhb,

  --Le cas d'induction (but : $\forall (b : mynat), a * b = a * \succ n → b = \succ n$) :
  intros b h, --On introduit un $b$ et l'hypothèse $h : a * b = a * \succ n$
  --Le but est juste $b = \succ n$ maintenant
  --On fait une distinction de cas sur $b$ :
  cases b with c,

  --Cas $b = 0$ (but : $0 = \succ n$) :
  --Cela contredit l'axiome de Peano $zero_ne_succ$, on va donc passer par l'absurde :
  rw mul_zero at h, --On simplifie $h$ pour obtenir $h : 0 = a * \succ n$
  exfalso,
  apply mul_pos a (succ n), --On a besoin de démontrer les hypothèses de mul_pos :
  --Hypothèse $a \ne 0$ :
  exact ha,
  --Hypothèse $\succ n \ne 0$ :
  exact succ_ne_zero n,
  --Retour à la preuve par l'absurde :
  symmetry,
  exact h,

  --Cas $b = \succ c$ (but : $\succ c = \succ n$) :
  repeat {rw succ_eq_add_one},
  rw add_right_cancel_iff, --On simplifie le but pour lui appliquer l'hypothèse d'induction
  --C'est là que le $revert \ b$ prend tout son importance car le but est $c = n$
  --On n'aurait pas pu appliquer l'hypothèse d'induction si elle prenait un $b$ particulier
  apply hn,
  --Il ne reste plus qu'a simplifier l'hypothèse $h$ :
  repeat {rw mul_succ at h},
  rw add_right_cancel_iff at h,
  exact h,
end
\end{minted}

\paragraph{Inequality World Level 15}

Dans la suite, nous allons définir > tel que :
\begin{minted}[mathescape,linenos,numbersep=5pt,frame=lines,framesep=2mm,breaklines,escapeinside=||]{lean}
	def a < b := a |$≤$| b |$∧ \lnot$| (b |$≤$| a)
\end{minted}
Mais la définition :
\begin{minted}[mathescape,linenos,numbersep=5pt,frame=lines,framesep=2mm,breaklines,escapeinside=||]{lean}
	def a < b := succ a |$≤$| b
\end{minted}
est plus pratique à utiliser et mathématiquement équivalente dans les entiers naturels.
Nous allons donc prouver que 
\begin{minted}[mathescape,linenos,numbersep=5pt,frame=lines,framesep=2mm,breaklines,escapeinside=||]{lean}
	(a b : mynat) : a |$≤$| b |$∧ \lnot$| (b |$≤$| a) → succ a |$≤$| b
\end{minted}
dans ce lemme (l'autre partie de l'équivalence est le niveau 16).

\begin{minted}[mathescape,linenos,numbersep=5pt,frame=lines,framesep=2mm,breaklines,escapeinside=||]{lean}
lemma lt_aux_one (a b : mynat) : a |$≤$| b |$∧ \lnot$| (b |$≤$| a) → succ a |$≤$| b :=
begin
  --On commence par transformer $a ≤ b ∧ \lnot (b ≤ a)$ en 2 hypothèses :
  intro h,
  cases h with hab htba,
  --On introduit $c$ tel que $b = a + c$
  cases hab with c hc,

  --On ne peut rien faire à ce niveau car le cas $b = 0$ pose problème
  --On fait donc une distinction de cas sur $b$ :
  cases b,

  --Cas $b = 0$ (impossible donc par l'absurde) :
  exfalso,
  apply htba,
  exact zero_le a,

  --Cas $b = \succ b$ (but : $\succ a ≤ \succ b$) :
  --Là, c'est le cas $a = 0$ qui nous pose problème
  cases a,

  --Cas $a = 0$ (trivial vu que $b != 0$) :
  apply succ_le_succ,
  exact zero_le b,

  --Cas $a = \succ a$ (but : $\succ (\succ a) ≤ \succ b$) :
  rw hc,
  rw succ_add,
  apply succ_le_succ, --but : $\succ a ≤ a + c$
  
  --Cette inégalité est impossible si $c = 0$
  cases c,

  --Cas $c = 0$
  rw add_zero at hc,
  exfalso,
  apply htba,
  use 0,
  rw add_zero,
  symmetry,
  exact hc,

  --Cas $c = \succ c$ (but : $\succ a ≤ a + \succ c$) :
  --Les +1 se simplifient
  rw add_succ,
  apply succ_le_succ,
  use c,
  refl,
end
\end{minted}
